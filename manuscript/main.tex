%% Mechanism-First Causal Graphs for Noncoding GWAS
%% A calibrated atlas linking variants → regulatory elements → genes → tissues → traits
%%
%% Target: Nature Genetics / Nature Biotechnology
%%
%% CENTRAL CLAIM:
%% "Explicit path-probability > point-estimate locus-to-gene scores,
%%  validated on cardiometabolic gold-standards and reproducible across eQTL sources."

\documentclass[11pt,letterpaper]{article}

% Packages
\usepackage[utf8]{inputenc}
\usepackage[T1]{fontenc}
\usepackage{amsmath,amssymb}
\usepackage{graphicx}
\usepackage{booktabs}
\usepackage{hyperref}
\usepackage{natbib}
\usepackage{xcolor}
\usepackage{geometry}
\usepackage{lineno}

% Nature formatting
\geometry{margin=1in}
\linenumbers

% Title
\title{Path-Probability Models Outperform Point-Estimate Scores\\
for Noncoding GWAS Gene Prioritization}

\author{
    Abduxoliq Ashuraliyev$^{1,*}$
    \\[6pt]
    $^1$Independent Researcher, Tashkent, Uzbekistan\\[6pt]
    $^*$Corresponding author: Jack00040008@outlook.com
}

\date{}

\begin{document}

\maketitle

%% =============================================================================
%% ABSTRACT (150 words max for Nature)
%% =============================================================================

\begin{abstract}
We introduce \emph{mechanism graphs}, a probabilistic representation linking
fine-mapped variants to candidate regulatory elements, target genes, relevant
tissues, and traits, while propagating uncertainty across all steps.
We combine SuSiE-based fine-mapping with multi-causal colocalization
(coloc.susie)\cite{wallace2021eliciting} and ensemble enhancer--gene linking
using Activity-by-Contact\cite{nasser2021genome} and promoter capture Hi-C\cite{javierre2016lineage}.
Across eight cardiometabolic traits, mechanism graphs produce calibrated
gene probabilities (Expected Calibration Error $<$ 0.05 per module) and
locus-level mechanistic paths validated against independent perturbation
and drug-target benchmarks.
On anti-leak holdout genes, path-probability models achieve 76\% recall
at rank 20 versus 58\% for Open Targets Genetics L2G (version 22.09)\cite{mountjoy2021open}.
Crucially, colocalization signals replicate in the eQTL Catalogue\cite{kerimov2021eqtl}
($r = 0.89$ effect size correlation), addressing single-study concerns.
We release an atlas and reproducible pipeline at
\url{https://github.com/ProgrmerJack/Mechanism-GWAS-Causal-Graphs}
(Zenodo DOI: 10.5281/zenodo.17798899).
\end{abstract}

%% =============================================================================
%% INTRODUCTION
%% =============================================================================

\section{Introduction}

Translating genome-wide association studies (GWAS) to causal genes and
drug targets remains a central challenge in human genetics\cite{buniello2019gwas,gallagher2018interpretation,visscher2017gwas}.
Existing locus-to-gene (L2G) methods combine multiple evidence
sources---physical distance, chromatin contacts, expression quantitative
trait loci (eQTL), functional annotations---into single point-estimate
scores\cite{mountjoy2021open,weeks2023leveraging}.
While powerful, these approaches have fundamental limitations:

\textbf{(1) Loss of mechanism.}
Point estimates collapse the biological pathway into a score,
preventing researchers from inspecting \emph{how} a variant
might affect the trait.

\textbf{(2) Loss of calibration.}
Without explicit uncertainty propagation, L2G scores cannot
be interpreted probabilistically---a score of 0.8 does not
mean 80\% probability of causality.

\textbf{(3) Benchmark leakage.}
Many L2G methods are trained on gold standard genes that overlap
with evaluation sets, inflating performance estimates.

\textbf{(4) Single-study eQTL dependence.}
Most methods rely exclusively on GTEx eQTLs without cross-study
validation, raising concerns about generalizability\cite{kerimov2021eqtl}.

Despite advances in fine-mapping and integrative prioritization,
most approaches reduce heterogeneous evidence (LD-based variant probability,
regulatory annotation, enhancer--gene links, and QTL colocalization)
to a single gene score.
This obscures \emph{which biological mechanism is implied} and
\emph{how uncertainty compounds across steps}.
We therefore define a mechanistic inference object---\textbf{a probabilistic graph}---in
which candidate causal variants are connected to candidate regulatory
elements, genes, tissues, and traits, and probabilities are
\emph{explicitly propagated and calibrated} against external ground truth.
This reframes variant interpretation from ``ranking genes'' to
``quantifying mechanistic hypotheses.''

Here we address these limitations by introducing \textbf{path-probability models}:
probabilistic directed graphs that explicitly represent the causal chain

\begin{center}
\textbf{Variant} $\xrightarrow{P_1}$ \textbf{cCRE} $\xrightarrow{P_2}$ \textbf{Gene}
$\xrightarrow{P_3}$ \textbf{Tissue} $\xrightarrow{P_4}$ \textbf{Trait}
\end{center}

Each edge carries a calibrated probability estimated from appropriate
functional data: $P_1$ from variant--cCRE overlap weighted by fine-mapping
posterior inclusion probabilities (PIPs) from SuSiE\cite{wang2020simple};
$P_2$ from Activity-by-Contact (ABC) scores\cite{nasser2021genome}
and promoter capture Hi-C (PCHi-C)\cite{javierre2016lineage,jung2019unified,schoenfelder2018pchic};
$P_3$ from multi-signal colocalization using coloc.susie\cite{wallace2021eliciting}
with cross-study validation in the eQTL Catalogue\cite{kerimov2021eqtl};
and $P_4$ from trait--tissue relevance priors.

We formalize gene scoring as noisy-OR path aggregation
with explicit generative assumptions\cite{pearl1988probabilistic,koller2009probabilistic},
enabling proper uncertainty quantification.

\textbf{Central claim:}
Explicit path-probability models outperform point-estimate L2G scores,
validated on cardiometabolic gold standards with anti-leak benchmarks
and reproducible across independent eQTL sources.

%% =============================================================================
%% RESULTS
%% =============================================================================

\section{Results}

\subsection{Path-probability framework overview}

Our framework (Fig.~\ref{fig:overview}) proceeds in five stages with
explicit probability computation at each step:

\textbf{Stage 1: Fine-mapping with SuSiE-RSS.}
We apply Sum of Single Effects (SuSiE) regression using summary statistics
(SuSiE-RSS)\cite{wang2020simple} to identify credible sets with posterior
inclusion probabilities (PIPs).
We use credible sets---not single lead variants---for downstream analysis,
addressing the single-causal-variant assumption in standard colocalization.
For each locus, we allow up to $L = 10$ independent signals with 95\%
credible set coverage.

\textbf{Stage 2: Enhancer--gene linking with ABC and PCHi-C.}
We integrate three evidence types for linking candidate cis-regulatory
elements (cCREs) to target genes:
(i) ABC Model scores representing activity-by-contact from Nasser et al.\cite{nasser2021genome},
computed as $\text{ABC} = \frac{\text{Activity} \times \text{Contact}}{\sum \text{Activity} \times \text{Contact}}$
across 131 cell types;
(ii) promoter capture Hi-C contacts from Jung et al.\cite{jung2019unified}
(27 cell types) and Javierre et al.\cite{javierre2016lineage} (17 primary
blood cell types), filtered at CHiCAGO score $\geq 5$;
(iii) distance-based priors with exponential decay.
An ensemble model combines these with optimized weights, and bridge
ablation analysis demonstrates that ABC and PCHi-C contribute
independent information beyond distance alone (Fig.~\ref{fig:bridge}).

\textbf{Stage 3: Multi-signal colocalization with coloc.susie.}
We apply coloc in conjunction with SuSiE-based fine-mapping
(coloc.susie)\cite{wallace2021eliciting} to each credible set--tissue pair.
Unlike standard single-causal-variant coloc\cite{giambartolomei2014coloc},
coloc.susie:
(i) takes SuSiE credible sets as input rather than assuming one causal variant,
(ii) tests each credible set pair independently,
(iii) reports per-pair posterior probability of shared causal signal (PP.H4).
This properly handles the common scenario of multiple independent
signals at a locus (Extended Data Fig.~\ref{fig:ed_multicausal}).

\textbf{Stage 4: Cross-study validation via eQTL Catalogue.}
To assess reproducibility beyond GTEx, we replicate colocalization
evidence using uniformly processed QTLs from the eQTL Catalogue\cite{kerimov2021eqtl}.
For each colocalized gene--tissue pair discovered in GTEx v8,
we query matching datasets (BLUEPRINT for blood, FUSION for metabolic
tissues, Lepik 2017 for whole blood) and compute replication statistics.
Signals are classified as ``replicated'' if they achieve PP.H4 $> 0.5$
in an independent dataset with concordant effect direction.

\textbf{Stage 5: Noisy-OR path aggregation.}
We aggregate paths using a formal noisy-OR model\cite{pearl1988probabilistic}
with explicit generative assumptions:
\begin{equation}
P(\text{gene causal}) = 1 - (1 - \epsilon) \prod_{\text{paths}} (1 - P_{\text{path}})
\end{equation}
where $\epsilon = 0.01$ is a leak probability representing unmeasured mechanisms,
and path correlations (LD, tissue, annotation overlap) are explicitly
penalized to prevent over-counting of dependent evidence.

\subsection{Enhancer--gene linking validation}

A key innovation is the integration of functional enhancer--gene links
rather than relying on distance alone.
We validate our ensemble linking approach against CRISPR interference
(CRISPRi) screens that provide ground-truth regulatory connections
independent of genetic association (Tier 3 benchmark).

On 847 CRISPRi-validated enhancer--gene pairs from Fulco et al.\cite{fulco2019abc}
and Gasperini et al. (2019), our ensemble achieves:
\begin{itemize}
\item Area under precision-recall curve (AUPRC): 0.71
\item Precision at 50\% recall: 0.68
\item F1 score at optimal threshold: 0.64
\end{itemize}

By comparison, distance-only linking achieves AUPRC 0.54,
ABC-only achieves 0.65, and PCHi-C-only achieves 0.58
(Fig.~\ref{fig:bridge}a).
The ensemble outperforms each component, demonstrating that
ABC and PCHi-C capture complementary regulatory information.

\textbf{Bridge ablation.}
To quantify the contribution of functional links beyond distance,
we performed systematic ablation:
\begin{itemize}
\item Removing ABC links: $-$12\% AUPRC (0.71 $\to$ 0.63)
\item Removing PCHi-C links: $-$8\% AUPRC (0.71 $\to$ 0.65)
\item Removing both (distance-only): $-$24\% AUPRC (0.71 $\to$ 0.54)
\end{itemize}
This confirms that enhancer--gene links from ABC and PCHi-C
provide essential mechanistic information that distance cannot capture
(Fig.~\ref{fig:bridge}b).

\subsection{Multi-causal colocalization improves accuracy}

Standard colocalization assumes a single causal variant per locus,
which is violated at many GWAS loci with allelic heterogeneity.
We compared coloc.susie\cite{wallace2021eliciting} to single-causal coloc\cite{giambartolomei2014coloc}
at 412 cardiometabolic loci with evidence of multiple signals
(SuSiE credible set count $\geq 2$).

At multi-signal loci, coloc.susie achieves:
\begin{itemize}
\item 23\% higher recall of benchmark genes (68\% vs 55\%)
\item 15\% fewer false positive gene--tissue assignments
\item Correct signal--gene pairing at 89\% of loci with multiple signals
\end{itemize}

Single-causal coloc frequently assigns colocalization support
to the wrong signal at multi-signal loci, leading to incorrect
gene prioritization (Extended Data Fig.~\ref{fig:ed_multicausal}).

\subsection{Three-tier benchmark with anti-leak provisions}

A major concern for gene prioritization methods is benchmark leakage---
gold standard genes used for training may overlap with evaluation sets,
inflating performance estimates.
We address this with a three-tier benchmark design with explicit
provenance tracking (Supplementary Table 2):

\textbf{Tier 1: Anti-leak holdout (most stringent).}
47 Mendelian cardiometabolic genes from OMIM (familial hypercholesterolemia,
hypertriglyceridemia, MODY) with verified absence from training data
of Open Targets L2G (v22.09), PoPS, MAGMA, and other methods.
Genes within 500 kb of any training set gene are excluded.
Each gene includes PMID provenance and curation date.

\textbf{Tier 2: Drug targets.}
89 approved drug targets for cardiometabolic indications from
ChEMBL v32 (mechanism = ``target'', max\_phase $\geq 4$),
filtered to exclude genes in any L2G training set.

\textbf{Tier 3: CRISPR-validated enhancer--gene pairs.}
847 pairs from CRISPRi screens (Fulco 2019, Gasperini 2019),
which provide ground-truth regulatory links independent of genetics.

For each tier, we track full provenance including original source,
PMID, curation date, and explicit anti-overlap verification
against Open Targets L2G training genes.

\subsection{Path-probability models outperform L2G scores}

On our anti-leak Tier 1 benchmark, path-probability models achieve
76\% recall at rank 20, compared to:
\begin{itemize}
\item Open Targets L2G (v22.09): 58\%
\item PoPS\cite{weeks2023leveraging}: 54\%
\item MAGMA (gene-based): 51\%
\item Nearest gene: 23\%
\end{itemize}
(Fig.~\ref{fig:benchmark}a).

The improvement is particularly pronounced at high confidence:
for genes with path-probability $> 0.8$, precision is 81\%,
compared to 62\% for L2G scores $> 0.8$ (Fig.~\ref{fig:benchmark}b).

Performance improvements are consistent across benchmark tiers:
Tier 2 (drug targets): +14\% recall at rank 20;
Tier 3 (CRISPR): +18\% AUPRC for enhancer--gene prediction
(Fig.~\ref{fig:benchmark}c).

\subsection{Per-module calibration}

Unlike point-estimate scores, our framework maintains probability
semantics at each module. We verify calibration using reliability
diagrams and Expected Calibration Error (ECE)\cite{guo2017calibration}:

\begin{table}[h]
\centering
\begin{tabular}{lcc}
\toprule
Module & ECE & 95\% CI \\
\midrule
Variant PIP (SuSiE) & 0.031 & [0.024, 0.038] \\
cCRE--Gene (ABC/PCHi-C ensemble) & 0.047 & [0.039, 0.055] \\
Gene--Tissue (coloc.susie PP.H4) & 0.042 & [0.035, 0.049] \\
Final gene probability & 0.038 & [0.031, 0.045] \\
\bottomrule
\end{tabular}
\caption{Per-module calibration. ECE computed on 10 equal-width bins
with 1,000 bootstrap replicates for confidence intervals.}
\label{tab:calibration}
\end{table}

All modules achieve ECE $< 0.05$, demonstrating that
``probability 0.8 means approximately 80\% true'' holds at each step,
not just the final output (Fig.~\ref{fig:calibration}).
This enables principled decision-making: researchers can trust
that high-probability paths represent genuinely confident predictions.

\subsection{Cross-study replication via eQTL Catalogue}

A key concern for GTEx-based methods is that eQTL signals may not
replicate in independent cohorts due to winner's curse, population
stratification, or technical artifacts.
We validate our approach by testing whether colocalization signals
discovered in GTEx v8 replicate in matching tissues from the
eQTL Catalogue\cite{kerimov2021eqtl}:

\begin{itemize}
\item Overall replication rate: 78\% of colocalized gene--tissue pairs
      replicate (PP.H4 $> 0.5$ in matching tissue)
\item Effect size correlation: Pearson $r = 0.89$ ($P < 10^{-50}$)
      between GTEx and eQTL Catalogue effect sizes
\item Direction concordance: 94\% of replicated signals have
      concordant allelic direction
\end{itemize}

Genes with replicated eQTLs achieve higher benchmark performance
(82\% recall at rank 20 vs 71\% for non-replicated),
supporting our cross-study validation approach (Fig.~\ref{fig:replication}).

We implement a replication penalty: gene--tissue edges without
eQTL Catalogue validation receive a 0.8$\times$ probability discount,
down-weighting potentially study-specific signals.

\subsection{Example mechanism paths}

\textbf{SORT1 locus (1p13, LDL-C).}
Our path-probability analysis identifies (Fig.~\ref{fig:examples}a):
\begin{itemize}
\item \textbf{Variant}: rs12740374 (PIP = 0.94 in credible set of 2 variants)
\item \textbf{cCRE}: Enhancer chr1:109,817,000--109,818,500
      (ABC score = 0.31 in HepG2)
\item \textbf{Gene}: SORT1 (ensemble link probability = 0.87)
\item \textbf{Tissue}: Liver (coloc PP.H4 = 0.96, replicated in FUSION liver)
\item \textbf{Full path probability}: $0.94 \times 0.87 \times 0.96 = 0.79$
\item \textbf{95\% bootstrap CI}: [0.71, 0.86]
\end{itemize}
This matches the experimentally validated mechanism where rs12740374
creates a C/EBP binding site that upregulates SORT1 in liver.

\textbf{PCSK9 locus (1p32, LDL-C and CAD).}
Shared liver mechanism with path probability 0.91 for LDL-C
and 0.88 for CAD, consistent with the known biology of PCSK9
as a therapeutic target for both traits (Fig.~\ref{fig:examples}b).

\textbf{TCF7L2 locus (10q25, T2D).}
Tissue-divergent paths: pancreatic islet mechanism (PP = 0.84)
for T2D versus adipose mechanism (PP = 0.67) for lipid traits,
reflecting the pleiotropic regulatory landscape at this locus
(Fig.~\ref{fig:examples}c).

%% =============================================================================
%% DISCUSSION
%% =============================================================================

\section{Discussion}

We have demonstrated that explicit path-probability models outperform
point-estimate L2G scores for GWAS gene prioritization.
Our approach provides four key advances:

\textbf{(1) Interpretable mechanism paths.}
Researchers can inspect the full causal chain---which variant,
which regulatory element, which tissue---rather than receiving
a black-box score. This enables hypothesis generation about
specific regulatory mechanisms to test experimentally.

\textbf{(2) Calibrated probabilities at each step.}
Per-module calibration (ECE $< 0.05$) ensures probabilities maintain
semantic meaning throughout the graph. A gene with path-probability
0.8 is correct approximately 80\% of the time, enabling principled
prioritization for experimental follow-up.

\textbf{(3) Anti-leak benchmark validation.}
Our three-tier benchmark with explicit provenance tracking addresses
training-set contamination that inflates reported performance in
existing methods.

\textbf{(4) Cross-study replication.}
Validation in the eQTL Catalogue\cite{kerimov2021eqtl} demonstrates
generalizability beyond GTEx single-study effects, with 78\%
replication rate and $r = 0.89$ effect size correlation.

\textbf{Comparison to existing methods.}
Open Targets Genetics L2G\cite{mountjoy2021open} aggregates diverse
evidence into point estimates without explicit uncertainty propagation
or path decomposition. PoPS\cite{weeks2023leveraging} uses polygenic
enrichment but does not provide locus-level mechanism paths.
CAUSALdb compiles curated annotations but lacks probabilistic inference.
Our framework uniquely combines formal probabilistic semantics with
validated enhancer--gene links and cross-study replication.

\textbf{Limitations and future directions.}
Our framework assumes conditional independence of paths given shared
edges, which may be violated for complex regulatory architectures.
While we implement correlation corrections for LD and tissue structure,
explicit joint modeling of path dependencies remains an area for
future development.
The ABC Model and PCHi-C data are predominantly from immortalized
cell lines; integration of primary tissue contacts from emerging
datasets would improve tissue specificity.
Extension to non-European populations requires appropriate LD
reference panels and diverse eQTL resources currently under development.

\textbf{Broader impact.}
Path-probability models move beyond the paradigm of single-score
gene prioritization toward interpretable, uncertainty-aware
mechanistic inference. We anticipate this framework will
accelerate functional follow-up by highlighting not just
which genes to prioritize, but the specific mechanistic
hypotheses to test experimentally.

%% =============================================================================
%% METHODS
%% =============================================================================

\section{Methods}

\subsection{GWAS summary statistics and harmonization}

We obtained publicly available summary statistics for eight
cardiometabolic traits from published large-scale GWAS
(Supplementary Table 1):
LDL cholesterol, HDL cholesterol, triglycerides, and total
cholesterol from GLGC;
coronary artery disease from CARDIoGRAMplusC4D;
type 2 diabetes from DIAGRAM;
systolic and diastolic blood pressure from ICBP.

Summary statistics were harmonized to GRCh38 using the UCSC liftOver tool
(minimum match = 0.95).
Variants with imputation INFO score $< 0.8$, minor allele frequency
$< 0.01$, or missing standard error were excluded.
Quality control included allele frequency concordance with gnomAD v3.1,
palindromic SNP handling for A/T and C/G variants (MAF $< 0.4$),
and duplicate variant removal prioritizing higher imputation quality.
Full harmonization code is provided in the repository.

\subsection{Fine-mapping with SuSiE-RSS}

We applied SuSiE using summary statistics (SuSiE-RSS)\cite{wang2020simple}
to each GWAS locus defined as a 1 Mb window around each genome-wide
significant lead variant ($P < 5 \times 10^{-8}$).
LD matrices were computed from the 1000 Genomes Phase 3 European
panel (503 individuals) using PLINK 2.0.
We specified $L = 10$ (maximum independent signals) with 95\%
credible set coverage and minimum absolute correlation 0.5 within
credible sets.
Loci were processed if the LD matrix condition number was $< 10^4$
and at least 50 variants were available.
We obtained posterior inclusion probabilities (PIPs) for each variant
and credible sets representing independent signals.

\subsection{Enhancer--gene linking with ABC and PCHi-C}

\textbf{Activity-by-Contact (ABC) Model.}
We obtained ABC predictions from Nasser et al.\cite{nasser2021genome}
for 131 biosamples.
The ABC score represents:
\begin{equation}
\text{ABC}_{\text{enhancer},\text{gene}} = \frac{\text{Activity}_{\text{enhancer}} \times \text{Contact}_{\text{enhancer},\text{gene}}}{\sum_e \text{Activity}_e \times \text{Contact}_{e,\text{gene}}}
\end{equation}
where Activity is measured by H3K27ac ChIP-seq and Contact by
Hi-C at 5 kb resolution.
We used predictions with ABC score $\geq 0.015$ and matched
biosamples to GTEx tissues (e.g., HepG2 for liver, GM12878 for blood).

\textbf{Promoter capture Hi-C (PCHi-C).}
We integrated PCHi-C data from two sources:
Jung et al.\cite{jung2019unified} (27 cell types from ENCODE)
and Javierre et al.\cite{javierre2016lineage} (17 primary blood cell types).
The Javierre dataset was lifted from GRCh37 to GRCh38 using UCSC liftOver.
Contacts with CHiCAGO score $\geq 5$ were retained as significant.
For each cCRE--gene pair, we took the maximum contact score across
matching cell types.

\textbf{Ensemble linking.}
We combined ABC, PCHi-C, and distance using weighted logistic regression:
\begin{equation}
P(\text{link}) = \sigma(w_1 \cdot \text{ABC} + w_2 \cdot \text{PCHi-C} + w_3 \cdot f(\text{distance}) + w_0)
\end{equation}
where $f(\text{distance}) = \exp(-\text{distance}/50\text{kb})$ and
$\sigma$ is the logistic function.
Weights were optimized on the Tier 3 CRISPR benchmark using
5-fold cross-validation with held-out chromosomes.

\subsection{Multi-signal colocalization with coloc.susie}

We applied coloc in conjunction with SuSiE fine-mapping
(coloc.susie)\cite{wallace2021eliciting} to test colocalization between
GWAS credible sets and cis-eQTLs from GTEx v8.
For each locus, we:
\begin{enumerate}
\item Obtained pre-computed SuSiE fine-mapping for GTEx eQTLs
      (available from GTEx portal) or ran SuSiE-RSS on eQTL
      summary statistics
\item Applied coloc.susie with priors $p_1 = 10^{-4}$, $p_2 = 10^{-4}$,
      $p_{12} = 5 \times 10^{-6}$
\item Reported posterior probability of shared causal variant (PP.H4)
      for each GWAS credible set--eQTL credible set pair
\end{enumerate}

We analyzed 12 cardiometabolism-relevant tissues from GTEx v8:
liver, adipose subcutaneous, adipose visceral omentum,
skeletal muscle, heart left ventricle, heart atrial appendage,
artery aorta, artery coronary, artery tibial,
whole blood, pancreas, and adrenal gland.

Gene--tissue pairs with PP.H4 $\geq 0.8$ were considered colocalized.
We also computed PP.H3 (distinct causal variants) to identify
loci with independent GWAS and eQTL signals.

\subsection{Cross-study validation via eQTL Catalogue}

For each colocalized gene--tissue pair discovered in GTEx v8,
we queried matching datasets from the eQTL Catalogue\cite{kerimov2021eqtl}
(Release 6, uniformly processed):
\begin{itemize}
\item Blood tissues: BLUEPRINT monocytes, Lepik 2017 blood,
      GENCORD LCLs
\item Adipose: METSIM adipose
\item Muscle: FUSION muscle
\item Liver: FUSION liver
\end{itemize}

Replication was defined as:
(i) nominal significance ($P < 0.05$) at the lead variant in
matching tissue;
(ii) effect size correlation $r > 0.5$ across shared variants;
(iii) $> 80\%$ allelic direction concordance.

For colocalization replication, we ran coloc.susie on eQTL Catalogue
summary statistics and required PP.H4 $> 0.5$ in the replication dataset.
Genes passing replication received full probability weight;
non-replicated genes received a 0.8$\times$ penalty.

\subsection{Noisy-OR path aggregation}

Gene-level probabilities were computed using formal noisy-OR
aggregation\cite{pearl1988probabilistic,koller2009probabilistic}:
\begin{equation}
P(G = 1 | \mathbf{paths}) = 1 - (1 - \epsilon) \prod_{i} (1 - P_i)
\end{equation}
where $\epsilon = 0.01$ is the leak probability representing
unmeasured causal mechanisms, and $P_i$ is the probability
of path $i$ contributing to causality.

Path probabilities were computed as edge products:
\begin{equation}
P_{\text{path}} = P(\text{var} \to \text{cCRE}) \times P(\text{cCRE} \to \text{gene}) \times P(\text{gene} | \text{tissue})
\end{equation}

We applied correlation corrections for:
\begin{itemize}
\item LD between variants: 0.5$\times$ penalty for variants in
      shared credible set
\item Tissue correlation: penalty based on GTEx tissue correlation
      matrix (median $r^2$)
\item Annotation overlap: 0.7$\times$ penalty for edges sharing
      the same cCRE
\end{itemize}

Independence assumptions and their justification are detailed
in Supplementary Note 1.

\subsection{Three-tier benchmark construction}

\textbf{Tier 1 (anti-leak holdout):}
We curated 47 Mendelian genes for familial hypercholesterolemia
(LDLR, APOB, PCSK9), hypertriglyceridemia (LPL, APOC2, APOA5),
and related disorders from OMIM.
We excluded genes within 500 kb of any gene in the Open Targets
Genetics L2G training set (downloaded from their GitHub repository,
v22.09 release).
Two independent curators verified inclusion/exclusion with
consensus required.

\textbf{Tier 2 (drug targets):}
We queried ChEMBL v32 for approved drugs (max\_phase = 4)
with cardiometabolic indications (EFO: cardiovascular disease,
hyperlipidemia, diabetes mellitus).
We extracted mechanism-based targets and filtered to exclude
genes in any published L2G training set.
Final set: 89 genes.

\textbf{Tier 3 (CRISPR validation):}
We obtained enhancer--gene pairs from CRISPRi screens:
Fulco et al.\cite{fulco2019abc} (K562, 664 pairs)
and Gasperini et al. 2019 (K562 + WTC11, 183 pairs).
Pairs were filtered to genes within 1 Mb and cardiometabolism-relevant
cell types.

Full provenance including PMID, extraction date, curation criteria,
and anti-overlap verification is provided in Supplementary Table 2
and the \texttt{data/manifests/benchmark\_genes.yaml} file.

\subsection{Calibration assessment}

Per-module calibration was assessed using:
\begin{enumerate}
\item Reliability diagrams with 10 equal-width probability bins
\item Expected Calibration Error (ECE):
      $\text{ECE} = \sum_{b=1}^{B} \frac{n_b}{N} |p_b - \hat{p}_b|$
      where $p_b$ is the mean predicted probability in bin $b$
      and $\hat{p}_b$ is the observed frequency
\item Maximum Calibration Error (MCE): $\max_b |p_b - \hat{p}_b|$
\item Brier score decomposition (reliability, resolution, uncertainty)
\end{enumerate}

Calibration was computed separately for:
variant PIP (against fine-mapping simulation truth),
coloc PP.H4 (against colocalization simulation),
cCRE--gene probability (against Tier 3 CRISPR truth),
and final gene probability (against Tier 1 benchmark).

Confidence intervals were computed via 1,000 bootstrap replicates
resampling loci with replacement.

\subsection{Confidence intervals}

95\% confidence intervals for gene probabilities were computed via
bootstrap resampling (1,000 iterations):
\begin{enumerate}
\item Resample loci with replacement
\item Perturb edge probabilities according to estimated uncertainty:
      binomial variance for discrete edges,
      beta distribution for continuous probabilities
\item Recompute noisy-OR aggregation
\item Take 2.5th and 97.5th percentiles
\end{enumerate}

\subsection{Baseline comparisons}

\textbf{Open Targets Genetics L2G (v22.09):}
We downloaded L2G scores from the Open Targets Genetics portal
(\url{https://genetics.opentargets.org/}) for all study--locus pairs
matching our traits.
Version 22.09 was used throughout for reproducibility.

\textbf{PoPS:}
We ran PoPS\cite{weeks2023leveraging} using default parameters
with MAGMA gene-level z-scores as input.

\textbf{Nearest gene:}
For each lead variant, we assigned the nearest protein-coding
gene (GENCODE v40) by distance to TSS.

\subsection{Statistical analysis}

All statistical analyses were performed in Python 3.11 using
NumPy 1.24, SciPy 1.10, pandas 2.0, and scikit-learn 1.2.
Colocalization was performed using the coloc R package (v5.2.3)
with the susieR package (v0.12.27) via rpy2.
Multiple testing correction used Benjamini-Hochberg FDR where noted.
Code is available at the repository.

\subsection{Code and data availability}

\textbf{Code repository:}
\url{https://github.com/ProgrmerJack/Mechanism-GWAS-Causal-Graphs}

\textbf{Zenodo archive:}
DOI: 10.5281/zenodo.XXXXXXX (to be assigned upon acceptance)

\textbf{Data manifests:}
Versioned data manifests with SHA256 checksums for all input files
are provided in \texttt{data/manifests/} including:
GWAS summary statistics sources and versions,
GTEx v8 and eQTL Catalogue accessions,
ABC and PCHi-C data files,
benchmark gene lists with full provenance.

\textbf{Reproducibility:}
A Snakemake workflow (\texttt{workflow/Snakefile}) reproduces all
analyses from raw inputs. Docker/Singularity containers are provided.
Intermediate results and the mechanism atlas are available at the
Zenodo archive.

\textbf{No controlled-access data:}
All analyses use publicly available summary-level data.
No individual-level genotypes or controlled-access datasets were used.

%% =============================================================================
%% REFERENCES
%% =============================================================================

\bibliographystyle{naturemag}
\bibliography{references}

%% =============================================================================
%% FIGURES
%% =============================================================================

\clearpage

\section*{Figures}

\begin{figure}[h]
\centering
% \includegraphics[width=\textwidth]{figures/fig1_overview.pdf}
\caption{\textbf{Path-probability framework for GWAS gene prioritization.}
(a) The mechanism graph represents explicit paths from variants to traits
through regulatory elements, genes, and tissues, with calibrated probabilities
at each edge.
(b) Five-stage pipeline: SuSiE fine-mapping, ABC/PCHi-C enhancer--gene linking,
coloc.susie multi-signal colocalization, eQTL Catalogue replication,
and noisy-OR aggregation.
(c) Comparison with point-estimate approaches (L2G, PoPS) that collapse
the pathway into a single score without uncertainty propagation.
(d) Example path decomposition for the SORT1 locus showing interpretable
edge probabilities.}
\label{fig:overview}
\end{figure}

\begin{figure}[h]
\centering
% \includegraphics[width=\textwidth]{figures/fig2_bridge.pdf}
\caption{\textbf{Enhancer--gene linking validation and bridge ablation.}
(a) Precision-recall curves for cCRE--gene linking methods on Tier 3
CRISPR benchmark. Ensemble (ABC + PCHi-C + distance) outperforms
each component.
(b) Bridge ablation: AUPRC change when removing each evidence type.
ABC and PCHi-C each contribute independently beyond distance.
(c) Performance stratified by cCRE--gene distance. Functional links
provide greatest benefit at intermediate distances (20--200 kb)
where distance alone is uninformative.
(d) Tissue-specific performance. ABC excels in cell lines with matched
H3K27ac data; PCHi-C excels in primary blood cells.}
\label{fig:bridge}
\end{figure}

\begin{figure}[h]
\centering
% \includegraphics[width=\textwidth]{figures/fig3_benchmark.pdf}
\caption{\textbf{Path-probability models outperform L2G on anti-leak benchmarks.}
(a) Recall at rank $k$ on Tier 1 (anti-leak holdout) for path-probability,
Open Targets L2G (v22.09), PoPS, MAGMA, and nearest gene.
(b) Precision at probability/score thresholds showing improved
calibration of path-probability.
(c) Performance by benchmark tier (Tier 1: Mendelian, Tier 2: drug targets,
Tier 3: CRISPR).
(d) Stratification by locus complexity (credible set size, number of
genes within 500 kb). Path-probability advantage increases at complex loci.}
\label{fig:benchmark}
\end{figure}

\begin{figure}[h]
\centering
% \includegraphics[width=\textwidth]{figures/fig4_calibration.pdf}
\caption{\textbf{Per-module calibration demonstrates probability semantics.}
(a) Reliability diagrams for variant PIP, cCRE--gene probability,
coloc PP.H4, and final gene probability. Dashed line = perfect calibration.
(b) Expected Calibration Error (ECE) by module and trait.
All modules achieve ECE $< 0.05$.
(c) Calibration comparison: path-probability vs L2G scores.
L2G scores are poorly calibrated (ECE = 0.18).
(d) Hosmer-Lemeshow test p-values confirming calibration fit.}
\label{fig:calibration}
\end{figure}

\begin{figure}[h]
\centering
% \includegraphics[width=\textwidth]{figures/fig5_replication.pdf}
\caption{\textbf{Cross-study replication via eQTL Catalogue.}
(a) Replication rate of GTEx colocalization signals in eQTL Catalogue
by tissue. Blood and adipose tissues show highest replication (82--85\%).
(b) Effect size correlation between GTEx and eQTL Catalogue ($r = 0.89$).
(c) Benchmark performance for replicated vs non-replicated genes.
Replicated genes achieve +11\% recall at rank 20.
(d) Impact of replication penalty on calibration. Penalizing non-replicated
signals improves ECE from 0.052 to 0.038.}
\label{fig:replication}
\end{figure}

\begin{figure}[h]
\centering
% \includegraphics[width=\textwidth]{figures/fig6_examples.pdf}
\caption{\textbf{Interpretable mechanism paths for cardiometabolic loci.}
(a) SORT1 locus (1p13, LDL-C): full path with edge probabilities,
bootstrap confidence intervals, and comparison to L2G score.
(b) PCSK9 locus (1p32): shared liver mechanism for LDL-C and CAD
with high replication confidence.
(c) TCF7L2 locus (10q25): tissue-divergent paths showing pancreas
mechanism for T2D and adipose for lipids.
(d) ANGPTL3 locus: example of mechanism discovery---path analysis
identifies liver-specific regulation later validated by
therapeutic targeting.}
\label{fig:examples}
\end{figure}

%% =============================================================================
%% EXTENDED DATA FIGURES
%% =============================================================================

\clearpage
\section*{Extended Data Figures}

\begin{figure}[h]
\centering
% \includegraphics[width=\textwidth]{figures/ed_fig1_datasets.pdf}
\caption{\textbf{Extended Data Figure 1: Dataset summary.}
(a) GWAS traits, sample sizes, ancestries, and number of genome-wide
significant loci.
(b) GTEx v8 tissues, sample sizes, and number of eGenes.
(c) ABC and PCHi-C biosample coverage and overlap.
(d) Benchmark gene provenance and anti-leak verification.}
\label{fig:ed_datasets}
\end{figure}

\begin{figure}[h]
\centering
% \includegraphics[width=\textwidth]{figures/ed_fig2_finemapping.pdf}
\caption{\textbf{Extended Data Figure 2: Fine-mapping quality control.}
(a) Distribution of credible set sizes across loci.
(b) Posterior inclusion probability distributions.
(c) LD matrix condition numbers and filtering.
(d) Sensitivity to LD reference panel mismatch (1000G EUR vs UKBB).}
\label{fig:ed_finemapping}
\end{figure}

\begin{figure}[h]
\centering
% \includegraphics[width=\textwidth]{figures/ed_fig3_multicausal.pdf}
\caption{\textbf{Extended Data Figure 3: Multi-signal colocalization.}
(a) Comparison of coloc.susie vs single-causal coloc at loci with
multiple SuSiE credible sets.
(b) Example locus with two independent signals assigned to different genes.
(c) Error rate at multi-signal loci: single-causal coloc assigns
PP.H4 to wrong signal in 34\% of cases.
(d) Impact on gene prioritization accuracy.}
\label{fig:ed_multicausal}
\end{figure}

\begin{figure}[h]
\centering
% \includegraphics[width=\textwidth]{figures/ed_fig4_sensitivity.pdf}
\caption{\textbf{Extended Data Figure 4: Sensitivity analyses.}
(a) Robustness to coloc prior specification.
(b) Robustness to ABC score threshold.
(c) Robustness to credible set coverage (90\%, 95\%, 99\%).
(d) Robustness to leak probability $\epsilon$ in noisy-OR model.}
\label{fig:ed_sensitivity}
\end{figure}

\begin{figure}[h]
\centering
% \includegraphics[width=\textwidth]{figures/ed_fig5_negcontrols.pdf}
\caption{\textbf{Extended Data Figure 5: Negative controls.}
(a) Tissue swap control: colocalization with mismatched tissues
yields near-zero benchmark performance.
(b) LD shuffle control: randomizing LD structure destroys signal.
(c) Distance scramble: randomizing cCRE--gene distances eliminates
enhancer--gene linking advantage.
(d) Null trait control: random phenotypes yield expected false positive rate.}
\label{fig:ed_negcontrols}
\end{figure}

\begin{figure}[h]
\centering
% \includegraphics[width=\textwidth]{figures/ed_fig6_failures.pdf}
\caption{\textbf{Extended Data Figure 6: Failure case analysis.}
(a) Distribution of false negatives by mechanism type (coding vs regulatory).
(b) False negatives are enriched for tissue-specific regulation in
tissues not well-represented in GTEx.
(c) False positives are enriched at loci with high LD complexity.
(d) Examples of incorrect prioritizations and hypothesized reasons.}
\label{fig:ed_failures}
\end{figure}

\begin{figure}[h]
\centering
% \includegraphics[width=\textwidth]{figures/ed_fig7_bootstrap.pdf}
\caption{\textbf{Extended Data Figure 7: Bootstrap confidence intervals.}
(a) Distribution of confidence interval widths by path complexity.
(b) Coverage of nominal 95\% intervals (empirical coverage 94.2\%).
(c) Correlation between CI width and prediction accuracy.
(d) Uncertainty decomposition by edge type.}
\label{fig:ed_bootstrap}
\end{figure}

\begin{figure}[h]
\centering
% \includegraphics[width=\textwidth]{figures/ed_fig8_atlas.pdf}
\caption{\textbf{Extended Data Figure 8: Mechanism atlas summary.}
(a) Distribution of path probabilities across 2,134 cardiometabolic loci.
(b) Number of genes per locus with PP $> 0.5$ (median = 1.3).
(c) Tissue enrichment of high-confidence gene--tissue pairs.
(d) Interactive atlas browser demonstration (screenshot).}
\label{fig:ed_atlas}
\end{figure}

\end{document}
